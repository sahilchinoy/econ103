\documentclass{article}
\usepackage{enumerate}
\usepackage{amsmath, amsthm, amssymb}
\usepackage[margin=1in]{geometry}
\usepackage[parfill]{parskip}
\DeclareMathOperator*{\argmax}{arg\,max}

\title{Econ C103 Problem Set 9}
\author{Sahil Chinoy}
\date{April 18, 2017}

\begin{document}
\maketitle{}

\subsection*{Exercise 1}

\begin{enumerate}[(a)]
	
	\item

	Suppose the mechanism did condition on the cardinal preferences of the agents. Take allocation $a$ to be the agent's most preferred alternative, so $u_i(a) \geq u_i(\hat{a})$ for $\hat{a} \in A$. To be incentive-compatible, the allocation must be monotone.  So the probability that the mechanism achieves allocation $a$ must be increasing in $m_i(a)$, the utility that the agent \textit{reports} for allocation $a$. Then, since the agent does not pay a transfer that depends on their report, it is preferable for the agent to report $m_i(a) \to \infty$. So the mechanism is not incentive compatible, since $m_i(a) \neq u_i(a)$.

	If instead the mechanism conditioned only on the ordinal preferences of the agent, the allocation would only change if the agent reported $m_i(a) < m_i(\hat{a})$ for some $\hat{a} \in A$ (because this implies that $a \prec_i \hat{a}$). But then, since the probability of allocation $a$ must be increasing in $m_i(a)$, this means the allocation $a$ occurs less often, and the agent is worse off. Then the agent is indifferent between any report $m_i(a) \geq u_i(a)$, thus the mechanism is dominant-strategy incentive compatible.

	So, to be dominant-strategy incentive compatible, a mechanism must condition only on the ordinal preferences of the agents. Note that this is necessary but not sufficient for dominant-strategy incentive compatibility.

	\item

	The set of all possible preferences is

	$$\{ \alpha \prec \beta \prec \gamma, \alpha \prec \gamma \prec \beta, \beta \prec \alpha \prec \gamma, \beta \prec \gamma \prec \alpha, \gamma \prec \alpha \prec \beta, \gamma \prec \beta \prec \alpha \}.$$

	\item

	Given $\alpha < \beta < \gamma$, the only single-peaked preferences are

	$$\{ \alpha \prec \beta \prec \gamma, \alpha \prec \gamma \prec \beta, \gamma \prec \alpha \prec \beta, \gamma \prec \beta \prec \alpha \}.$$


	\item

	No, agent 3's preferences are not single-peaked. In the median voting mechanism, allocation $\beta$ is implemented.

	\item

	Yes, if agent 3 reports $\alpha$, then the allocation will be $\alpha$, which the agent prefers to $\beta$.

\end{enumerate}

\end{document}